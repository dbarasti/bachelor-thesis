%acronimi
\newacronym{crs}{CRS}{Centro Ricerca e Sviluppo}
\newacronym{smi}{SMI}{Sanmarco Informatica}
\newacronym{mes}{MES}{manufacturing execution system}
\newacronym{bu}{BU}{business unit}
\newacronym{rtc}{RTC}{Rational Team Concert}
\newacronym{ide}{IDE}{Integrated Development Environment}
\newacronym{tdd}{TDD}{Test Driven Development}
\newacronym{wamp}{WAMP}{Web Application Messaging Protocol}
\newacronym{rpc}{RPC}{Remote Procedure Call}

%definizioni
\newglossaryentry{outsourcing}{name={outsourcing},description={L'appalto a una società esterna di determinate funzioni o servizi, o anche di interi processi produttivi.}}
\newglossaryentry{business unit}{name={business unit},description={Un sottoinsieme dell'impresa che rappresenta un business specifico concentrato su una particolare linea di prodotti.}}
\newglossaryentry{digital industry}{name={digital industry},description={Relativamente all'industria 4.0, ci si riferisce al concetto di fabbriche dove le macchine sono potenziate con connettività \textit{wireless} e sensori, connesse a un sistema che può tenere sotto controllo l'intera filiera produttiva e compiere delle decisioni sulla base dei dati raccolti autonomamente.}}

\newglossaryentry{rational team concert}{name={Rational Team Concert},description={Sviluppato da IBM, è una tecnologia per il team management che centralizza le principali funzionalità fondamentali per un team di sviluppo software: versionamento del codice, issue tracking system e sistema di build del codice.}}

\newglossaryentry{websocket}{name={WebSocket},description={Protocollo di comunicazione che permette di comunicare in maniera \textit{full-duplex}, cioè a due direzioni, su una singola connessione TCP.}}

\newglossaryentry{driver}{name={driver},description={Implementazione di uno specifico protocollo di comunicazione. È racchiuso in un componente ed espone un'interfaccia che rende semplici agli utenti operazioni come connessione, disconnessione e invio richieste.}}


\newglossaryentry{polling}{name={polling},description={Attività che viene effettuata ripetutamente per verificare il cambiamento di un certo stato. Nell'ambito del progetto l'attività di polling è quella effettuata nei confronti dei PLC per rilevare cambiamenti nei dati al suo interno.}}


\newglossaryentry{benchmark}{name={benchmark},description={}}