%!TEX root = ../dissertation.tex
\chapter{Retrospettiva}\label{ch:retrospettiva}

\section{Soddisfazione degli obiettivi}

Al termine delle 344 ore svolte presso Sanmarco Informatica è stato possibile ricapitolare quanto svolto e determinare lo stato di completamento degli obiettivi posti ad inizio stage.\\

Sia gli obiettivi obbligatori che quelli desiderabili sono stati portati a termine. Nella tabella seguente si trova il riassunto di quanto svolto:

\begin{center}
	\begin{longtable}{||p{7cm}| p{2.5cm}| p{2cm} ||} 
		\hline
		Obiettivo & Tipologia & Stato \\ [0.5ex] 
		\hline\hline
		Analisi dell'attuale infrastruttura e delle problematiche che hanno portato alla scelta di modificare l’architettura & Obbligatorio & Completato\\ 
		\hline
		Analisi omnicomprensiva della nuova architettura & Obbligatorio & Completato\\
		\hline
		Studio di fattibilità in termini "tecnologici" (ossia reperimento e/o sviluppo di componentistica
		necessaria alla realizzazione) e relativi \gls{benchmark} di confronto
		& Obbligatorio & Completato \\
		\hline
		Sviluppo prototipo al fine di testare le soluzioni trovate e le performance effettive delle stesse (in termini di tempo di esecuzione, scalabilità e risorse impegnate) & Desiderabile & Completato \\ [1ex]
		\hline
		\caption{Riassunto obiettivi obbligatori e desiderabili con stato di completamento}
		\label{tab:soddisfacimento_obiettivi}
	\end{longtable}
\end{center}

La suddivisione delle ore preventivate nel piano di lavoro redatto prima dell'inizio dello stage ha rispecchiato quasi tutti i punti previsti. Di seguito sono riassunte le ore impiegate per lo svolgimento delle attività.

\begin{itemize}
	\item Formazione sulle tecnologie: 40 ore. Per questa attività è stato fondamentale il contatto con i membri del team JMES. Questi mi hanno aiutato nell'adattamento alle nuove tecnologie e nella comprensione del \textit{framework} Scrum in cui ho lavorato;
	
	\item Analisi (vecchia e nuova struttura): 104 ore. Poter ricevere i consigli e le motivazioni di chi in prima persona aveva portato avanti lo sviluppo ha reso l’attività di analisi della struttura esistente più efficiente. Questo mi ha permesso di risparmiare alcune ore che ho dedicato alle attività successive;
	
	\item Realizzazione PoC: 56 ore. La realizzazione del \acrshort{poc} mi ha consentito non solo di creare una bozza concreta del progetto ma anche di affinare i requisiti raccolti. Con il supporto del tutor SMI molti di essi sono infatti stati rielaborati. I \textit{benchmark} realizzati sul PoC hanno inoltre verificato che i vincoli posti durante l’analisi fossero raggiungibili;
	
	\item Realizzazione prototipo: 144 ore. Una volta fissati i requisiti è stato possibile organizzare l’architettura di JDI. Determinati i componenti principali sono passato alla codifica. Fondamentale è stato il \textit{feedback} da parte del tutor per fissare man mano le diverse parti.
		
\end{itemize}

Le attività di formazione e realizzazione PoC sono risultate perfettamente in linea rispetto alla pianificazione. L’attività di analisi è invece stata sovrastimata. Questo mi ha permesso di dedicare l’equivalente di tre giorni in più alla realizzazione del prototipo. 


\section{Sviluppi futuri}

Un punto fermo per il tutor Alex prevedeva che il prototipo da me realizzato fosse predisposto ad essere successivamente esteso, per sostituire definitivamente la precedente versione di JDI sviluppata da \acrshort{smi}.\\
Nell'ultima settimana di stage si sono susseguite diverse riunioni per determinare il proseguo dello sviluppo di JDI. Il riassunto di queste è stato che entro settembre 2019 si sarebbe dovuti arrivare alla conclusione della codifica delle funzionalità mancanti. Entro la fine dell'anno è invece prevista la prima installazione presso il cliente che per primo aveva segnalato i problemi del primo sviluppo sottolineati nel capitolo \ref{ch:problematiche}.

Il prodotto illustrato in questo documento ha quindi raggiunto una maturità tale da poter essere impiegato dal team JMES, con cui sono sempre stato a stretto contatto, nella continuazione dello sviluppo.

I successivi passi da svolgere per il team in modo da ottenere un prodotto pronto ad essere installato in produzione consisteranno in:

\begin{itemize}
	\item Creazione dell'interfaccia \textit{web} di JDI che permetterà di condividere i dati raccolti con la famiglia di prodotti \acrshort{smi}, in primis JMES;
	\item Integrazione dei numerosi \gls{driver} esistenti per interagire con una vasta gamma di PLC che altri potenziali clienti utilizzano.
\end{itemize}

\section{Considerazioni finali}

Lo stage svolto presso \acrlong{smi} mi ha trasmesso molto più di quanto sia possibile riassumere in un documento di poche pagine. Sono stato inserito in un team che quotidianamente affronta problemi su larga scala, sia umani che tecnici.\\

Quello che mi ha spinto a cogliere la sfida posta da questo stage è stata la possibilità di lavorare non solo con il software ma anche con l’hardware. In questo senso il diploma in Automazione, conseguito prima di iniziare il percorso di laurea triennale, mi ha fornito le basi necessarie per comprendere la terminologia che i tecnici SMI utilizzavano e gli strumenti che non sono usuali nell'informatica tangibile nel nostro corso di studi.


Ho apprezzato particolarmente gli insegnamenti che il triennio mi ha dato. Ho potuto mettere in pratica le conoscenze acquisite soprattutto nei corsi di Ingegneria del Software, Programmazione Concorrente e Distribuita e Tecnologie Open Source.
Ho percepito il mio livello di preparazione sufficiente da poter affrontare gli argomenti proposti potendo sempre aggiungere un mio pensiero critico.\\
Volendo invece trattare le lacune che ho riscontrato, avrebbe sicuramente giocato a mio favore la presenza nel piano di studi di corsi affini al mondo industriale. Comprendendo che ciò ricade solitamente nella parte ingegneristica dell'informatica, credo che fornire le conoscenze di base su un argomento come quello dei controllori a logica programmabile non possa che arricchire ed incuriosire lo studente.\\

Terminata questa esperienza, confrontandomi con amici, compagni di corso, ho realizzato l’abissale differenza nei progetti svolti. Le tecnologie impiegate sono infatti diametralmente opposte. Mi sono scontrato con problemi che si hanno quando si lavora ad un basso livello di astrazione. Gran parte dei componenti realizzati non facevano uso di librerie esterne. La gestione della concorrenza è stata fatta a basso livello, forzandomi a comprendere fino in fondo quanto stavo sviluppando. Nessun \textit{framework} ha indirizzato il mio lavoro.\\
Credo che questa esperienza sia stata utile per riconoscere la complessità del lavorare con tecnologie di base, ma ha anche provato che non è sempre necessario affidarsi al codice di terzi per realizzare un prodotto, soprattutto se si richiede il totale controllo di ciò che si utilizza. In questo caso il software prodotto è risultato più difficile da scrivere ma ha raggiunto gli obiettivi di manutenibilità posti inizialmente. \\

Ringrazio quindi \acrlong{smi} e il team JMES per avermi permesso di vivere un'esperienza che ha arricchito il mio curriculum con le migliori pratiche di sviluppo software ma sopratutto che mi ha insegnato quanto importanti siano le relazioni interpersonali nell'ambiente di lavoro.


