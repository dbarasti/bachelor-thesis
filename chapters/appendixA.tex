%!TEX root = ../dissertation.tex
\chapter{Nota sui PLC}
\label{ch:appendice_plc}

PLC sta per \textit{Programmable Logic Controller}, controllore a logica programmata. Si tratta di un computer ideato per il mondo industriale. Controlla differenti processi ed è programmato in base ai requisiti del processo a cui viene applicato.

Molte industrie mettono in pratica processi produttivi specifici, ad esempio per la creazione di un certo bene. Modificare questo processo richiede il rifacimento di gran parte dell'apparato produttivo utilizzato, ad un costo estremamente elevato.

Per superare questo problema una prima versione del PLC fu inventata da \textit{Dick Morley}, che al tempo lavorava per \textit{Modicon}, nel 1968 \footnote{[https://library.automationdirect.com/history-of-the-plc/]}. Un PLC può essere in breve descritto come un sistema di controllo che contiene la definizione di una sequenza programmata.