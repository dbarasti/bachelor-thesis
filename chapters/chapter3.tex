%!TEX root = ../dissertation.tex
\begin{savequote}[75mm]
La prima versione di JDI era organizzata a nodi \textit{router-link} che potenzialmente potevano permettere di distribuire il prodotto in zone fisiche
diverse, comunicando poi tramite \textit{\gls{websocket}}.
Nella realtà però gli \textit{stakeholder} di JMES hanno constatato che gli svantaggi introdotti
dalla soluzione sviluppata superavano possibili vantaggi commerciali del prodotto.
\end{savequote}

\chapter{Le Problematiche}\label{ch:problematiche}
In questo capitolo si analizzerà brevemente la situazione che ha portato al primo sviluppo di JDI nel 2018, gli errori commessi durante lo sviluppo e la progettazione, il sorgere dei primi problemi arrivati in produzione e i propositi per la nuova versione di JDI, oggetto della tesi.

\section{Il primo sviluppo di JDI}

Dai bisogni sottolineati in \ref{jdi} \acrshort{smi} ha deciso di iniziare lo sviluppo di un \textit{middleware} (JDI) che fosse in grado di interfacciarsi con i PLC utilizzando i giusti protocolli di comunicazione. Lo sviluppo iniziò a maggio 2018 e venne assegnato alla \acrshort{bu} JMES. Terminò circa sei mesi dopo.

Per sottolineare ulteriormente la separazione logica tra JMES e JDI è utile ricordare che JMES serve all'operatore per determinare come organizzare il proprio turno lavorativo. JDI invece mira ad aiutare l'operatore nelle operazioni manuali che possono facilmente portare a errori. Ad esempio la dichiarazione dei pezzi prodotti da un macchinario o la segnalazione dell'interruzione di funzionamento di una macchina e di tutte le operazioni \textit{error-prone}.
\newline

Lo sviluppo è stato eseguito da un solo programmatore a cui non erano stati posti vincoli precisi per la progettazione dell'applicazione, se non l'utilizzo del linguaggio Java. Questo ha dato piena libertà sulla scelta delle tecnologie di sviluppo.

Architetturalmente JDI si presentava nel seguente modo: un software a servizi con compatibilità Windows. I servizi rappresentavano nodi che potevano essere \textit{Router} o \textit{Link}. Ogni nodo veniva avviato come servizio Windows. I \textit{Link} erano servizi che utilizzavano un file di configurazione per effettuare la connessione con i PLC. Ognuno stabiliva una connessione con i PLC utilizzando il protocollo adatto (ricavato dal file di configurazione) e una connessione verso il nodo \textit{Router} tramite \textit{\gls{websocket}}. Quest'ultima serviva a comunicare i risultati delle operazioni svolte dai nodi \textit{Link} su i PLC. Nella figura \ref{fig:nodi_jdi} sono rappresentate queste connessioni.

Il nodo \textit{Router} effettuava delle chiamate http verso JMES per comunicare le variazioni dei dati rilevati dai nodi \textit{Link}. A quali dati fosse interessato JMES era specificato nei file di configurazione che ogni link utilizzava per determinare le richieste da fare ai PLC.

Le richieste erano principalmente
\begin{itemize}
	\item Lettura di registri \textit{general purpose} su cui il PLC manteneva dati relativi al funzionamento della macchina a cui era collegato. Ad esempio il numero di pezzi prodotti fino a quel momento;
	\item Scrittura di registri \textit{general purpose}, per resettare determinati stati, o per attivare/disattivare funzioni sulla macchina.
\end{itemize}

La soluzione sviluppata aveva un primo visibile vantaggio: la struttura a nodi poteva essere distribuita in punti fisici diversi. L'importante era che la comunicazione tra nodi \textit{Link} e \textit{Router} potesse avvenire. Questo era garantito dalla connessione \textit{WebSocket} stabilita, che era indipendente dalla posizione dei nodi.
\newline

\label{fig:nodi_jdi}
\begin{figure}[h]
	\centering
	\includegraphics[width=0.8\textwidth]{figures/nodi_jdi}
	\caption{Architettura a nodi \textit{Router}/\textit{Link} della prima versione di JDI}
\end{figure}



\section{I problemi con JDI}
La soluzione sviluppata presupponeva un utilizzo di JDI che facesse uso della possibilità di distribuire i suoi nodi su più punti. Nella realtà però gli \textit{stakeholder} di JMES hanno constatato che le esigenze dei clienti non prevedevano lo sfruttamento di questa caratteristica. Gli svantaggi che come vedremo ha portato la soluzione sviluppata erano molto maggiori dei possibili vantaggi commerciali del prodotto.
\newline

A seguito dello sviluppo, sono iniziati i primi test del prodotto. Questi test dovevano verificare la stabilità del sistema e sopratutto la sua affidabilità nel lungo termine. Una volta installano in ambiente di produzione sarebbe infatti dovuto rimanere attivo per lunghi periodi, facilmente oltre l'anno. \\
I test prevedevano invece sessioni troppo brevi e con un numero di dati coinvolti
nella trasmissione non realistici. Non è mai stata effettuata un'analisi
dinamica sul consumo di risorse.\\

I primi problemi sono emersi in produzione, quando dopo pochi giorni di utilizzo il sistema presentava dei cali prestazionali e delle interruzioni impreviste.

A seguito di questi eventi avvenuti presso la sede di un cliente sono intervenuti i tecnici sistemisti di JMES che hanno tentato di effettuare il \textit{debug} del codice scoprendo numerosi problemi legati all'utilizzo di librerie esterne che rendevano inoltre difficoltoso analizzare le zone del codice in cui si verificavano degli errori.

Le problematiche emerse hanno sottolineato carenze che si riassumono in:
\begin{itemize}
	\item basse prestazioni (velocità di trasmissione);
	\item scarsa affidabilità (stabilità del sistema a lungo termine);
	\item alto consumo di risorse;
	\item bassa manutenibilità.
\end{itemize}

%Si discute della mancanza di test di affidabilità e di scalabilità del prodotto sviluppato, dell'inutilità della possibilità di distribuire fisicamente JDI e dei drawback a cui l'architettura scelta portava e della scarsa manutenibilità del codice.\\
%Di questi problemi ci si è accorti solo arrivati in produzione, cioè quando JDI è stato installato nelle fabbriche di alcuni clienti.




\section{Il nuovo sviluppo}

Data l'esperienza precedente, la nuova versione di JDI, oggetto di questa tesi, è stata sottoposta a vincoli più rigidi. 

A differenza della versione precedente, che avviava un servizio Windows per ogni nodo, il programma doveva essere eseguibile su un \textit{web server} Apache Tomcat, doveva limitare in consumo di risorse impiegate sopratutto dopo lunghi periodi di esecuzione e poter gestire un numero ragionevolmente alto di flussi di dati.

Inoltre visto lo scarso successo della versione distribuita di JDI, la nuova versione doveva sfruttare la concorrenza su singola JVM per adempiere ai compiti di gestione dei diversi flussi da e verso i PLC.

Un importante punto per JDI è che potesse essere facilmente manutenibile e che facesse uso il meno possibile di librerie esterne che hanno reso la manutenzione complicata 56nella prima versione.\\

Lo scopo in pratica era di raggruppare le funzionalità di base che vecchia versione offriva sotto forma di servizi Windows, in un ambiente \textit{multi-threadded} e:
\begin{itemize}
	\item garantirne la scalabilità;
	\item valutare quantitativamente le sue caratteristiche, attraverso \textit{benchmark} di prototipi;
	\item rendere disponibile i dati ricavati con servizi web. Principalmente a JMES ma in generale a chi ne possa trarre vantaggio;
	\item storicizzare i dati raccolti.
\end{itemize}
Gli ultimi due punti sono obiettivi più a lungo termine per JDI, che non rientrano nel dominio dello stage svolto.


