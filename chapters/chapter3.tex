%!TEX root = ../dissertation.tex
\begin{savequote}[75mm]
JDI era organizzato a nodi \textit{router-client} che potenzialmente potevano permettere di distribuire il prodotto in zone fisiche
diverse, comunicando poi tramite \textit{WebSocket}.
Nella realtà però gli \textit{stakeholder} di JMES hanno constatato che gli svantaggi introdotti
dalla soluzione sviluppata superavano possibili vantaggi commerciali del prodotto.
\end{savequote}

\chapter{Le Problematiche}\label{ch:problematiche}
In questo capitolo si analizzerà brevemente la situazione che ha portato al primo sviluppo di JDI, gli errori commessi durante lo sviluppo e la progettazione, il sorgere dei primi problemi arrivati in produzione e i propositi per la nuova versione di JDI, oggetto della tesi.

\section{Il primo sviluppo di JDI}
L'idea di ridurre gli errori commessi durante l'interazione operatore-JMES tramite l'utilizzo di un middleware in grado di comunicare con i PLC e che effettuasse il recupero di dati che l'operatore inseriva solitamente manualmente (e.g. la conta dei pezzi prodotti da una macchina, la segnalazione di un guasto o di un fermo, ecc).

Si parla poi dello sviluppo della prima versione: poche indicazioni sulle tecnologie, nessuna restrizione sui requisiti non funzionali ecc.\\
Inoltre si discute della scelta architetturale che avrebbe permesso di distribuire i nodi di JDI ovunque nel mondo.




\section{I problemi con JDI}
Si discute della mancanza di test di affidabilità e di scalabilità del prodotto sviluppato, dell'inutilità della possibilità di distribuire fisicamente JDI e dei drawback a cui l'architettura scelta portava e della scarsa manutenibilità del codice.\\
Di questi problemi ci si è accorti solo arrivati in produzione, cioè quando JDI è stato installato nelle fabbriche di alcuni clienti.




\section{Il nuovo sviluppo}

Gli obiettivi che il nuovo JDI deve raggiungere, il cambiamento di architettura che puntava a eliminare la distribuzione del codice e che prediligeva la concorrenza.\\
Le garanzie di affidabilità e scalabilità ecc.