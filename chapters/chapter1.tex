%!TEX root = ../dissertation.tex
\begin{savequote}[75mm]
L’avvento e la crescita di internet hanno rivoluzionato il modo di fare impresa in particolare quella industriale. 
Ciò che prima era un impianto isolato, con operatori  e responsabili, ora è parte di una catena produttrice interconnessa che genera non solo beni tangibili ma anche dati che, se ben gestiti, possono diventare informazione utilizzata per aumentare l’efficienza dell'intera impresa in un delicato ciclo di feedback.
\end{savequote}

\chapter{Introduzione}

Prima di affrontare il problema oggetto di questa tesi è doveroso fornire una panoramica del contesto aziendale che ha caratterizzato lo stage svolto. In questo modo i capitoli successivi risulteranno più chiari e inseriti in una propria logica.

\section{Sanmarco Informatica SpA}\label{smi}

L’azienda \acrfull{smi} dal 1984 si occupa di consulenza e sviluppo \textit{software}. Si è specializzata nella progettazione, realizzazione e installazione di soluzioni a supporto dei processi aziendali di duemila aziende con numerose installazioni oltre confine.\\
L’azienda ha tra i suoi punti di forza quello della vicinanza ai clienti, che si traduce in diverse sedi di proprietà in Veneto, Lombardia, Emilia-Romagna e Friuli-Venezia Giulia con 450 dipendenti totali.\\

Il \acrfull{crs} è situato a Grisignano di Zocco (VI), sede di lavoro per oltre 150 dipendenti. Questo è il fulcro dello sviluppo e della manutenzione dei prodotti \textit{software}. Team di sviluppatori e sistemisti sono coordinati per garantire stabilità di servizio per i clienti, e un’alta personalizzazione dei prodotti offerti.\\
La ricerca e l’innovazione sono di grande valore per \acrshort{smi}, con una media del 20\% di fatturato investito annualmente in ricerca e sviluppo.
Sono anche numerose le risorse impiegate nella formazione e ricerca di talenti provenienti dall'Università di Padova. Un esempio di questo è il progetto \textit{Academy} che viene descritto nel seguente modo: "Lo scopo è quello di educare giovani allievi da inserire direttamente nel mondo lavorativo, attraverso un percorso di eccellenza che forma figure professionali altamente qualificate."\footnote{\href{www.sanmarcoacademy.com}{www.sanmarcoacademy.com}}.\\
Il \acrshort{crs} è anche responsabile per l’erogazione dei servizi \textit{cloud} forniti da \acrshort{smi}. È infatti presente una sala server amministrata da tecnici sistemisti.\\
Tutta la proprietà operativa di \acrshort{smi} rimane in azienda, senza il bisogno di \textit{\gls{outsourcing}} ad aziende esterne.\\

L’azienda è organizzata in \textit{\acrfull{bu}}, un sottoinsieme dell'impresa che rappresenta un \textit{business} specifico concentrato su una particolare linea di prodotti.\\

\begin{wrapfigure}{r}[1pt]{0.25\textwidth}
	\centering
	\includegraphics[width=0.25\textwidth]{figures/logo_jmes}
	\caption{Logo \acrshort{bu} Jmes}
\end{wrapfigure}
La {\acrshort{bu} interessata dallo stage è JMES, composta da 20 persone tra sistemisti e sviluppatori, che si occupa dell'omonimo applicativo utilizzato nell'ambito della \textit{\gls{digital industry}}. Questo software serve a gestire e supervisionare l'avanzamento della produzione all'interno della fabbrica.\\
%\begin{wrapfigure}{l}{0.3\textwidth}
%	\caption{Albero delle \gls{business unit} di \acrshort{smi}}
%	\centering
%	\includegraphics[width=0.3\textwidth]{figures/albero_prodotti_smi}
%\end{wrapfigure}

Altre \acrshort{bu} si occupano invece di \textit{Business Process Management} (JPA), gestione contenuti aziendali come manuali e documenti della qualità (Discovery ECM), \textit{marketing} (nextBI), sviluppo di applicazioni e siti per aziende (4words) e molto altro.
\begin{figure}[h]
	\includegraphics[width=0.3\textwidth]{figures/albero_prodotti_smi}
	\centering
\end{figure}

Ulteriori informazioni riguardanti le \acrshort{bu} o l'azienda in generale possono essere trovate sul sito ufficiale di  \href{https://www.sanmarcoinformatica.com}{Sanmarco Informatica}.

%\clearpage

\section{La Digital Industry}
Introduzione alla sezione qui?

\subsection{L'industria 4.0}
L'avvento e la crescita di internet hanno rivoluzionato il modo di fare impresa in particolare quella industriale. \\
Ciò che prima era un impianto isolato, con operatori  e responsabili, ora è parte di una catena produttrice interconnessa che genera non solo beni tangibili ma anche dati che, se ben gestiti, possono diventare informazione utilizzata per aumentare l’efficienza dell'intera impresa in un delicato ciclo di \textit{feedback}.\\
La corretta gestione di questa informazione diventa fondamentale per il successo dell'impresa poichè mantiene alta la competitività.

Questi nuovi aspetti vengono spesso riassunti nel termine \textit{quarta rivoluzione industriale}, un altro termine comune in questo ambito è infatti \textit{Industry 4.0}.

\begin{figure}[h]
	\centering
	\includegraphics[width=0.9\textwidth]{figures/rivoluzioni_industriali}
	\caption{Una visione temporale delle rivoluzioni industriali fino ad oggi}
\end{figure}

Nasce quindi la necessità di un software che trasformi il dato in informazione e che faccia uso di questa per migliorare le \textit{performance} supportando e gestendo a sua volta tutta la linea di produzione con indicazioni precise e in tempo reale.\\
Questo strumento prende il nome di \textit{\acrfull{mes}}. \\
Un software \acrshort{mes} in pratica si occupa di raccogliere, elaborare e condividere informazioni provenienti direttamente dall'industria che normalmente restano disaggregate per diversi motivi: si trovano in formato non elettronico, sono poco precise o si trovano in ambienti isolati tra loro.

\subsection{Manufacturing execution system}\label{mes}


Il controllo e la gestione dell'informazione consentono non solo di mantenere un piano generale del processo produttivo aziendale, ma anche di evidenziare possibili problemi in anticipo e ottenere quindi una migliore resa di produzione.\\
Vediamo quali sono i principali benefici di un sistema \acrshort{mes}\footnote{[\textit{Five benefits of a MES}], \href{https://www.industryweek.com/companies-amp-executives/five-benefits-mes}{www.industryweek.com/companies-amp-executives/five-benefits-mes}}:
\begin{enumerate}
	\item \textbf{Riduzione di scarti e sprechi}. Grazie ad una visione in tempo reale della produzione, è possibile individuare con piccoli margini situazioni in cui le unità prodotte non sono conformi, fermando la produzione e limitando gli sprechi;
	\item \textbf{Precisione nella definizione dei costi}. Con un sistema \acrshort{mes}, i tempi di lavoro, gli sprechi, i tempi di inattività, la manutenzione sono monitorati e registrati in tempo reale direttamente dalla fabbrica. Questo rende innanzitutto l’informazione più affidabile, e utilizzabile per riclassificare i costi e valutare criticamente situazioni con alto tasso di sprechi;
	\item \textbf{Ridurre i tempi di inattività}. Siccome una produzione ferma difficilmente porta a un guadagno, un \acrshort{mes} deve essere in grado di pianificare la produzione in modo che le risorse necessarie siano quelle disponibili, integrando anche i piani dei turni dei dipendenti, in modo da aumentare ulteriormente l’efficienza, riducendo quindi i costi;
	\item \textbf{Riduzione del magazzino}. Le giacenze in magazzino costano, perché aumentano i costi logistici di gestione, oltre ai costi per produrre tali giacenze. Con un \acrshort{mes} si ha uno stato aggiornato della nuova produzione, degli scarti e dei prodotti non conformi. In questo modo chi si occupa di acquisto, spedizione e pianificazione sa esattamente cosa è disponibile e cosa bisogna ordinare;
	\item \textbf{Riduzione di costi}. Con un controllo più stretto sui tempi e i costi necessari per la produzione, è possibile snellire ulteriormente processi a supporto, come la gestione logistica del magazzino, quindi ottenendo un disimpegno di persone e attrezzature.
	
\end{enumerate}

\section{Le squadre JMES}

Come accennato nel paragrafo \ref{smi} esistono due squadre che lavorano in maniera sinergica per lo sviluppo e l’installazione del prodotto JMES:
\begin{itemize}
	\item \textbf{Sviluppatori}. Sono raggruppati all'interno del team di JMES e si occupano di implementare le funzionalità aggiuntive richieste dai clienti. Le richieste possono provenire da un singolo cliente o essere inserite in distribuzioni di aggiornamento per tutte le installazioni. La differenza sta nel valore monetario dell' \textit{update}. La figura a cui fanno riferimento è lo \textit{Scrum Master}. Ad agosto 2019 il team è composto da otto persone;
	\item \textbf{Sistemisti}. Effettuano un’analisi tecnica presso le sedi dei clienti evidenziando possibili problemi di compatibilità con le attrezzature presenti, propongono le diverse configurazioni del software JMES in base a requisiti e vincoli posti dal cliente, organizzano l’installazione e la configurazione del prodotto. La figura a cui fanno riferimento è il capo progetto. Ad agosto 2019 il gruppo è composto da dieci persone.
\end{itemize}

Per il periodo dello stage sono stato inserito nel team di sviluppo JMES.\\

La prima installazione di JMES risale a maggio 2018. È quindi un prodotto giovane che però è stato già apprezzato da 19 clienti attivi, ci sono inoltre 27 analisi per l’installazione in corso e altre 19 da pianificare.

\clearpage

\section{Metodologia di lavoro}
\subsection{Sviluppo agile e framework Scrum}
Indipendentemente dalle \acrshort{bu} aziendali, la metodologia di lavoro per la gestione del ciclo di  sviluppo del software è Agile, implementata con il \textit{framework} Scrum.

Agile è una disciplina per lo sviluppo di software che pone al primo posto tra i suoi principi la soddisfazione e il coinvolgimento del cliente e la distribuzione di software funzionante in maniera regolare e a distanza di brevi periodi, dalle due settimane ai due mesi.

Il \textit{framework} Scrum prevede di suddividere un periodo di lavoro, chiamato \textit{sprint cycle} in tre fasi principali:
\begin{itemize}
	\item \textit{Planning}: il team comunica con gli \textit{stakeholder} (rappresentati dal \textit{product owner}), analizza e comprende i requisiti creando lo \textit{sprint backlog}, composto di requisiti che il prodotto deve soddisfare entro la fine dello \textit{sprint}. Questi prendono il nome di \textit{story} se sono completabili entro uno sprint. Un insieme correlato di \textit{Story} si chiamano \textit{Epic};
	\item \textit{Implementation}: durante questa fase dello \textit{sprint} il team crea delle porzioni complete di prodotto. Le funzionalità implementate in uno \textit{sprint} provengono dallo \textit{sprint backlog}. La durata di questa fase è, nel caso del team JMES, quattro settimane;
	\item \textit{Review}: ci si riunisce per revisionare il lavoro svolto e pianificare ciò che non è stato possibile portare a termine nella fase di \textit{implementation}. Ogni membro del team mostra una \textit{demo} delle funzionalità sviluppate nella fase precedente;
	\item \textit{Retrospective}: vengono analizzate in modo retrospettivo le fasi precedenti in un’ottica di miglioramento continuo dei processi al fine di renderli più efficienti negli sprint successivi. Si discute di strumenti e metodologie utilizzate e di come queste abbiano influito nello sviluppo. Se vengono ritenute poco utili, la loro pratica viene dismessa.
\end{itemize}



\subsection{Tecnologie di sviluppo del team}

%Per lo sviluppo dell'applicazione è stato utilizzato il linguaggio Java. \\
%Per l'interfacciamento con i PLC è stato utilizzato il protocollo ModBusTCP in un'implementazione proprietaria di \acrshort{smi}.\\
Descritte le tecnologie di sviluppo del team JMES. \\
Principalmente si paralerà del linguaggio Java e del framework sviluppato da \acrshort{smi} per lo sviluppo: \textit{Synergy}.



\subsection{Strumenti di lavoro del team}
Al centro degli strumenti per la gestione del flusso di lavoro del team JMES c'è uno strumento sviluppato da IBM, \acrfull{rtc}. Questo offre le seguenti funzionalità:
	\begin{itemize}
		\item versionamento del codice;
		\item gestione dei \textit{work item} provenienti dallo \textit{sprint backlog} (\textit{issue tracking system});
		\item \textit{build tool}.
	\end{itemize}
\acrshort{rtc} si integra all'interno dell'ambiente di sviluppo del team grazie a un \textit{plug-in} completo per l'\acrshort{ide} Eclipse;\\

\clearpage

\section{Outline del documento}

L'obiettivo di questo documento è di raccogliere e tradurre a parole l'esperienza fatta durante lo stage curricolare presso Sanmarco Informatica svolto nel periodo 10 giugno 2019 - 8 agosto 2019.

A partire dal prossimo capitolo si toccheranno tutti gli argomenti tecnici trattati. \\
Iniziando con un chiarimento sulle tecnologie adottate per il progetto di stage, si passerà poi alla motivazione che ha spinto \acrshort{smi} a fare questa proposta di stage. Infine nel capitolo \ref{ch:analisi_progettazione_sviluppo} verranno sintetizzate le attività di analisi, progettazione e sviluppo che sono iniziate a fine giugno 2019 e sono terminate agli inizi di agosto 2019.\\

Nonostante il tempo limitato, durante lo stage ho cercato di esplorare diversi aspetti dell'ingegneria del software che durante il percorso triennale ho potuto analizzare solo dal punto di vista teorico.\\
L'esperienza mi ha infatti permesso di trascorrere del tempo in un team che utilizzava la disciplina \textit{Agile}, ho sperimentato la tecnica del \textit{pair programming} durante una fase dello sviluppo e ho potuto mettere in uso il modello del \textit{\acrfull{tdd}}, anche in questo caso in maniera limitata rispetto all'intero progetto.\\


Nel capitolo \nameref{ch:background} vengono analizzate più nel dettaglio le tecnologie già citate in questo capitolo, come il software JMES. Sono introdotti i linguaggi di programmazione utilizzati e gli strumenti per lo sviluppo e la gestione del codice. \\

Il capitolo \nameref{ch:problematiche} intende presentare i problemi che hanno portato alla proposta di stage. Il prodotto sviluppato durante lo stage mira a riprodurre le funzionalità principali di un software già presente in \acrshort{smi}, il cui sviluppo è terminato a metà 2018, che presentava diversi problemi sul piano architetturale. Questi hanno minato la stabilità e l'affidabilità del prodotto, che necessitava quindi di un rifacimento.\\

Nonostante il fallimento della prima versione di JDI, la seconda versione da me sviluppata doveva riprodurre le funzionalità di base per poter essere una solida base di partenza per il resto del team JMES che una volta terminato lo stage avrebbe preso in carico il progetto. Il capitolo \ref{ch:analisi_progettazione_sviluppo} riassume l'analisi della prima versione e la raccolta dei requisiti. L'attività di progettazione ha ricoperto un ruolo particolarmente importante poichè doveva evitare gli errori commessi durante la progettazione del primo JDI. Verrà qui messa in luce la contrapposizione tra l'architettura distribuita della prima versione e quella concorrente della seconda.\\

Infine nel capitolo \ref{ch:retrospettiva} raccolgo le impressioni dell'esperienza di stage all'interno di \acrshort{smi}, di quale preparazione ho sentito la mancanza e per quali aspetti ho più apprezzato il corso di studi.



