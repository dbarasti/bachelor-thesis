%!TEX root = ../dissertation.tex
\begin{savequote}[75mm]

\end{savequote}

\chapter{Background tecnologico}
In questo capitolo si fornirà una introduzione alle tecnologie che risulterà utile per comprendere al meglio i capitoli successivi.\\
Gli argomenti trattati saranno principalmente JMES, JDI e alcuni standard della digital industry (e.g. protocollo Modbus)


\section{JMES}

Nella sezione \ref{mes} sono stati individuati i principali benefici che l'utilizzo di un \textit{software} \acrshort{mes} porta.\\
Vediamo quali sono i benefici che JMES, l'implementazione MES di \acrshort{smi} intende portare ai clienti con il suo prodotto: 
\begin{enumerate}
	\item Avanzamento processi produttivi. L’informazione sull'avanzamento della produzione è disponibile in tempo reale, permettendo una gestione più flessibile del lavoro. Ad esempio gli uffici commerciali che si trovano in sede distaccata dal centro produttivo possono avere informazioni dettagliate sullo stato di avanzamento di un ordine in breve tempo, senza nemmeno interpellare gli operatori o i responsabili di produzione;
	\item Gestione materiali. La disponibilità in magazzino dell'ordine completato è immediatamente rilevata, permettendo un proseguimento di processo più reattivo. Si pensi ad esempio alla possibilità di richiedere una spedizione appena l’ordine risulta saldato; 	
	\item \textit{Monitoring} dei processi. Oltre all'avanzamento del singolo ordine, di interesse specialmente per operatori e impiegati, è possibile effettuare il monitoraggio al fine di supportare la \textit{business intelligence} aziendale, cioè l’insieme delle strategie usate dall'impresa per analizzare i dati di produzione (stato dell'impianto, individuazione dei problemi, macchinari che portano spesso a rallentamenti); 
	\item Consuntivazione. Il beneficio per cui i sistemi \acrshort{mes} nascono: rilevare i tempi di processo per valutare il discostamento da quanto preventivato. Impiegare più tempo del previsto significa ridurre il margine di guadagno che all'estremo può portare ad una perdita. Rilevare gli scostamenti permette di arrivare principalmente a due conclusioni:
	\begin{itemize}
		\item Lo standard di valutazione è errato, ottenendo dalla serie storica dei rilevamenti una conferma che il processo produttivo non è in grado di rispettare i tempi preventivati;
		\item Il processo è migliorabile. Ci sono delle casistiche che portano ad una degradazione occasionale dei tempi di produzione. Ad esempio fermi macchina ricorrenti o operatori non abbastanza efficienti.
	\end{itemize}	
	La possibilità di rilevare queste informazioni permette di accorgersi in tempo di     situazioni critiche in cui ad esempio i rilevamenti effettuati si allontanano dal piano di budget;
	\item Progetto carta zero. Grazie alla gestione software della produzione, si riduce la carta circolante che molto spesso veniva utilizzata per tracciare le fasi degli impianti, per registrare le interazioni operatore-macchina e per le stampe di documenti tecnici di assemblaggio.
\end{enumerate}

Parlare ora più nel dettaglio di JMES.

\clearpage

\section{Gli standard della Digital Industry}
\subsection{ModBus}

\clearpage


\section{L'interazione macchina-MES: JDI}
Con un sistema \acrshort{mes} tradizionale esistono due tipi di interazione: uomo-macchina e uomo-\acrshort{mes}. La prima consiste nell'insieme di azioni che l’operatore svolge sulla macchina per avanzare nel processo produttivo, la seconda consiste nella dichiarazione delle azioni svolte dall'operatore al sistema \acrshort{mes}. 

Questa approccio può generare un bias che consiste nella differenza tra quanto effettivamente svolto durante la prima interazione e quanto dichiarato nella seconda in quanto l’operatore può accidentalmente compiere degli errori durante le dichiarazioni manuali.\\
Per questo il mercato ha richiesto una soluzione in grado di permettere una maggiore precisione delle rilevazioni di produzione. \\
L’elaborazione di questo bisogno ha portato a considerare l’introduzione di una terza interazione ai fini di ridurre quando possibile l’interazione operatore-\acrshort{mes}. 

Alle macchine industriali moderne è quasi sempre collegato un PLC che si occupa di comandare la macchina a cui e connesso ed elabora e immagazzina i dati che vengono generati, relativi ad esempio al numero di pezzi prodotti, lo stato degli allarmi o l'interruzione di funzionamento della macchina.

La nuova interazione è integrata in un prodotto a se stante, indipendente da JMES, chiamato JDI. L'idea alla base di questo prodotto è un software che si interfacci con i PLC nelle aziende dei clienti per intercettare i dati immagazzinati.

Questi dati sono accuratamente gestiti e resi disponibili ad altre applicazioni della famiglia \acrshort{smi}.

\begin{figure}[h]
	\includegraphics[width=1\textwidth]{figures/plc-jdi-jmes}
	\caption{Rappresentazione di come JDI si integra nei sistemi di produzione e con JMES}
	\centering
\end{figure}

\clearpage