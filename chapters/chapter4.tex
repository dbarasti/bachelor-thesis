%!TEX root = ../dissertation.tex
%\begin{savequote}[75mm]
%\qauthor{Quoteauthor Lastname}
%\end{savequote}

\chapter{Analisi, Progettazione e Sviluppo}\label{ch:analisi_progettazione_sviluppo}

\section{Studio della versione precedente}
Nonostante il primo sviluppo di JDI non avesse avuto un risultato positivo una volta messo in produzione, le funzionalità di base di questa dovevano essere presenti anche nella nuova versione.\\
Come visto nel capitolo precedente, i problemi stavano nella scarsa soddisfazione di requisiti non funzionali, non nel soddisfacimento di quelli funzionali.

Parlerò poi dell'uso di librerie esterne fatto in questa versione (e.g. Google Guice, ReactiveX...)

In questa sezione si sottolineerà l'approccio di distribuzione della vecchia versione. Nelle seguenti sezioni invece apparirà chiaro come la distribuzione sia stata messa da parte sostituendola con un puro approccio concorrente.

Sarà quindi descritta la fase di analisi durata circa due settimane. 
 

\section{I requisiti del progetto}
In questa sezione vengono raccolti i requisiti estrapolati dall'analisi della vecchia versione.

Particolare attenzione sarà posta sui requisiti non  funzionali. Come detto sono stati il punto debole della versione precedente.


\section{Metodologia di lavoro}
Way of working utilizzato durante il progetto, interazione con il team JMES di cui ero parte ecc..

\section{Tracer bullet development}
Definire questo progetto un prototipo sarebbe errato, infatti spesso questi sviluppi servono solo a dimostrare la fattibilità di un progetto e il codice utilizzato viene quasi sempre scartato.\\
Il lavoro svolto invece è stato mirato alla creazione e all'implementazione di un'architettura solida, scalabile e affidabile della nuova versione di JDI, l'opposto quindi di un prototipo! \\
Questo tipo di sviluppo mira a utilizzare la maggior percentuale possibile di tecnologie per stendere una prima strada verso lo sviluppo completo.
Questa metodo prende il nome di \textit{tracer bullet development}.


Si parlerà principalmente di: 
\begin{itemize}
	\item lavoro di integrazione iniziale delle tecnologie e degli strumenti utilizzati durante lo sviluppo. 
	\item la progettazione di prototipi e l'analisi di prestazione con benchmark per testare le prestazioni prima ancora di iniziare lo sviluppo intensivo;
	\item la scelta del prototipo
	\item lo sviluppo 
\end{itemize}

